
\documentclass[12pt]{scrartcl}

\input{../styles/Packages.tex}

\begin{document}

\section{Themen}
	\begin{enumerate}
	
	\item Eigenschaften von Sprachen
		\begin{enumerate}
			\item Chomsky Hierarchie 
				\begin{figure}[!h]
				\centering
				\includegraphics[width=1\textwidth]{Pictures/Chomsky.png}
				\includegraphics[width=1\textwidth]{Pictures/Verhältnisse.png}
				\end{figure}
			\item Definiton Grammatik (Ableitungen) 
			\item Abschlusseigenschaften
				\begin{figure}[!h]
				\centering
				\includegraphics[width=0.9\textwidth]{Pictures/Abschlusseigenschaften.png}
				\end{figure}
			\item Typische Vertreter einer Sprache
			\item 
		\end{enumerate}
		
	\item Automaten
		\begin{enumerate}
			\item NEA und DEAS
			\item Potenzmengenkonstruktion
			\item Abschlusseigenschaften
			\item Kellerautomat (det / non det) - Konstruktion aus Sprache 
			\item Turingmaschinen 
		\end{enumerate}
		
	\item Grammatiken
		\begin{enumerate}
			\item Typ 3 Sprache - Linksableitung/Syntaxbäume
			\item Reguläre Ausdrücke
			\item Chomsky Normalform
			\item Greibach Normalform
			\item Turingmaschine zu Grammatik
			\item Typ 1 Sprachen, Kuroda-Normalform
		\end{enumerate}
		
		\item Mathematischer Teil
		\begin{enumerate}
			\item Erkennung durch Monoide 
			\item Myhill-Neurode
			\item syntaktische Monoid + Beweismethode für Typ 3
			\item allg. Beweis das eine Sprache eines best Types ist 
			\item Pumping Lemma Typ 3
			\item Pumping Lemma Typ 2
		\end{enumerate}
\end{enumerate}

\newpage

\section{Lösungen der Übungsaufgaben}
	

\end{document}
