
\documentclass[12pt]{scrartcl}

\input{../styles/Packages.tex}

\newcommand\circlearound[1]{%
  				\tikz[baseline]\node[draw,shape=circle,anchor=base] {#1} ;}
  
\begin{document}
	
\section{Theo I}

\subsection{Theo-Ka 18/19}

\subsubsection{Aufgabe 1}
	\begin{enumerate}
		\item \( (a|b)^{*}(aa|bb) \)
		\item Hat 5 Zustände
		\item (q0)-c->(q1)-a,b->(q3)-a,b->(q4) //machts besser
	\end{enumerate}
	
\subsubsection{Aufgabe 2}
	\begin{enumerate}
		\item
		\item
		\item \( [\varepsilon], [a], [ba] \)
		\item nein zb baab ist nicht in der Sprache aber bbbb schon
	\end{enumerate}
	
\subsubsection{Aufgabe 3}
	\begin{enumerate}
		\item 
		\begin{enumerate}
			\item \(\{\varepsilon\}\)
			\item \{$\varepsilon$,ab, aba, abaa \}
			\item \(\{ b^{*}a^{*}\}\)
			\item \( \Sigma^{*}\)
		\end{enumerate}
		\item Angenomen wir haben einen deterministischen endlichen Automaten der L erkennt. Seien Z die Menge der Zustände und E die Menge der Endzustände. Falls \( q \in Z \land \delta(q,a) \in E  \). Soll q auch als Endzustand markiert werden. Dies wird nun so oft wiederholt, bis keine neuen Endzustände hinzugefügt werden.\\
		Sei w = va$^{n}$ und w $\in$ L.  Dann soll gezeigt werden, dass v akzeptiert wird.  \\ 
		
		Induktionsanfang: w = v = va$^{0}$, dann erkennt der oben konstruierte Automat v, da v = w $\in$ L und der konstruierte Automat offensichtlich alle Elemente in L erkennt().\\ 
		
		Es gilt: w = va$^{n-1}$, v \(\in multipop_a(L)\). Es ist zu zeigen w = v'a$^{n}$, v'\(\in multipop_a(L)\). \\ 
		Aus w = va$^{n-1}$ = va$^{n}$ ergibt sich v = v'a. \\ Der konstruierte Automat der v akzeptiert muss einen Übergang \( \delta(q, v) \in E \) besitzen. \(\delta(q, v) \in E =\delta(q, v'a) \in E = \delta(\delta(q,v), a) \in E \Rightarrow \delta(q,v) \in E\) (aus Konstruktion).\\
		
		Zu zeigen ist noch, dass keine anderen Wörter erkannt werden als in Multipop enthalten sind.
		\qed	
	\end{enumerate}

\subsubsection{Aufgabe 4}
	\begin{enumerate}
		\item Falsch - zB \(\{a^{n}b^{n}|n \in \mathbb{N}\} \cap \{b^{n}c^{n}|n,k \in \mathbb{N}\} = \{ \}\)
		\item Falsch. Gegenbeispiel Wähle L = \(\{ a^{n}b^{n} | n \ge 0 \}\) und den homomorphismus f(a) = a, f(b) = $\varepsilon$ dann ist f(L) = a$^{*}$. Somit ist L nicht regulär aber das Bild unter dem homomorphismus. 
		\item Wahr.Typ 2 Sprachen unter Vereinigung abgeschlossen. Beweis mit Induktion. (Vereinige 0 mal) und dann n+1 mal. // Fragwürdig, stimmt dies? 
	\end{enumerate}
	
\subsubsection{Aufgabe 5}
	\begin{enumerate}	
		\item \(\{S \Rightarrow aS|aS',\ S' \Rightarrow bS'c|bc \}\)
		\item //
		\item Wir nehmen an L wäre regulär. Reguläre Sprachen sind unter homomorphismen abgeschlossen. Wir wählen f(a) = $\varepsilon$, f(b) = b, f(c) = c. f(L) = \(\{b^{n}c^{n} | n \ge 1\}\)\\ 
		Pumping Lemma sei x $\in$ L, x = uvw, wobei |x| $\ge$ n, |uv| $\le$ n, |v| $\ge$ 1. Sei x = \(b^{n}c^{n}\). \\
		Daraus ergibt sich, dass v = b$^{k}$. k > 1. \\
		Wähle i = 2: \(|uv^{2}w|_b = n + k > n = |uv^{2}w|_c \Rightarrow uv^{2}w \notin L \Rightarrow f(L) \) nicht regulär $\Rightarrow$ L nicht regulär .
		\item DEA = NEA $\subset$ DPDA $\subset$ PDA $\subset$ LBA $\subset$ DTM = TM
		\end{enumerate} 

\subsubsection{Aufgabe 6}
	\begin{enumerate}	
		\item Ist in Chomsky- und Kuroda-Normalform und vom Typ 0, 1 , 2 
		\item 
			\begin{table}[!h]	
			\centering
 			 \begin{tabular}{|c|c|c|c|}
 			 	 \cline{1-4}
     			 A,C & A,C & B & B \\ \cline{1-4}
     			 C & E &   \\ \cline{1-3}
     			  & B\\ \cline{1-2}
     			 E\\ \cline{1-1}
			\end{tabular}	
			\end{table} 
			Das Wort ist nicht in der Sprache. Da es nicht abgeleitet werden kann, ausgehend von dem Terminalzeichen S 
		\item Diese ist mehrdeutig, da es mehrere Ableitungsbäume gibt. S -> CD -> CBA -> aba \\
		S -> EC -> ABC -> aba
		\item 
	
\subsubsection{Aufgabe 7}
	\begin{enumerate}
			\item \{$\varepsilon$\}
			\item \{aab, aba, baa\}
			\item \{cba, cab, acb, abc, bac, bca\}
		\end{enumerate}
		\item Nein. Zum Beispiel L=a$^{*}$ = s(L)
	\end{enumerate}
	
\subsubsection{Aufgabe 8}
	\begin{enumerate}
			\item (aBbaa)-(abAaa)-(abaBa)-(abCaa)-(aCaaa)-(Caaaa)-(C\_aaaa)-(Aaaaa) \\
			Die Turing maschine bleibt nicht stehen. Das heißt das Wort wird nicht erkannt.
			\item (ab)$^{*}$
			\item M' = \{Q, $\Sigma$, $\Sigma \cup \{_\}$, M' \}
	\end{enumerate}
	
\subsection{Theo-Ka 17/18}
	
\subsubsection{Aufgabe 2}
	Wir betrachten x = a$^{4{^n}}$ = uvw. Wobei |uv| $\le$ n, |v| $\ge$ 1. \\
	Sei i = 2:\\
	uvw < uv$^{2}$w = a$^{4^{n}+k}$ <  a$^{4^{(n+1)}}$. \\
	Da 4$^{n}$ streng monoton wachsend ist uv$^{2}$w $\notin$ L $\Rightarrow$ L nicht regulär \\
	\qed

\subsubsection{Aufgabe 4}
	\begin{enumerate}
		\item \(G = \{ S \Rightarrow aS| aS',\ S' \Rightarrow bS'c|bc\}\)
		\item
		\item L ist nicht regulär. Dies ist mit Hilfe der paarweisen in Äquivalenz zu zeigen.\\
		Wir betrachten die Äquivalenzklassen der Sprache L \(\{ [ab^{b^n+1}c]| n \ge 1 \}\). Zu zeigen ist die paarweise Inäquivalenz dieser. Sei k > 0\\
		(1)\( [ab^{1}c] \neq [ab^{1+k}c]\), da \(ab^{1}c \in L\) aber \(ab^{1+k}c \notin L\) \\
		\( [ab^{b^n}c] \neq [ab^{b^n+k}c]\), da\( ab^{b^n}cc^{n-1} \in L\) aber \( ab^{b^n+k}cc^{n-1} \notin L\). Somit ist jede Äquivalenzklasse Inäquivalent zueinander 
	\end{enumerate}
	
\subsubsection{Aufgabe 4}
	\begin{enumerate}
		\item abaa
		\item aaabb
		\item \(\{aaaa^{n}b^{n}|n\in \mathbb{N}\}\)
		\item unendlich viele 
	\end{enumerate}
	
\subsubsection{Aufgabe 6}
	\begin{enumerate}
	
		\item Die Sprache ist regulär, da \(L_1 = \{w | w \in a^{+} \lor w \in b^{+} \} \)
		\item Die Sprache ist regulär, da \( L_2 = \{c^{*} \} \)
		\item Die Sprache ist det kontextfrei da der kellerautomat nicht raten muss 
		\item Die Sprache ist Typ 1 da sie \( a^{n} b^{n} c^{n} \) ähnelt
		\item Die Sprache ist Typ 2, da sie \(a^{n} b^{n} \) ähnelt jedoch muss sie raten, ob sie die gleichheit von a's und  bs's oder die Gleichheit von a's und c's überprüfen muss
		\item Die Sprache ist Typ 1, da ein endlicher nicht- deterministische Automat diese erkennen kann (wie ww)
		\item Diese ist det. kontextfrei (vgl a$^{n}$b$^{n}$)
		\item Diese ist Typ 2, da Typ 2 (a$^{n}$b$^{n}$ a$^{m}$b$^{m}$)) erkennen kann, aber raten muss wo b$^{m}$ anfängt
		\item Diese ist Typ 1 da unär und nicht Typ 3
		\item Diese ist Typ 3
	\end{enumerate}

\subsubsection{Aufgabe 7}
	\begin{enumerate}
		\item G = \(\{S \rightarrow S'abba | ,\ S' \rightarrow S'AS'BS'| S'BS'AS'|\varepsilon \}\)
	\end{enumerate}

\subsubsection{Aufgabe 8}
	\begin{enumerate}
		\item falsch, wahr, wahr, \(L_3=\{ b^{m+2} a^{n+1}| m,n \in \mathbb{N}_0 \}  \)
		\item Sei das Alphabet $\Sigma$ = \(\{a_1, a_2, ..., a_n  \}\) n $\in$ $\mathbb{N}\ge 1$. Wir ersetzen jedes Terminal a$_n$ in der Grammatik G mit einer Variable A$_n$ und Vereinigen diese mit\(\{ (a_k,A_k A_k),(A_k,a_k)| 0 < k < n \}\). Die resultierende ist G'.
	\end{enumerate}
	
\subsubsection{Aufgabe 9}
	\begin{enumerate}
		\item Ja, Nein, Nein, Nein
		\item 2
		\item Ja, Nein, Nein, Nein
		\item S $\Rightarrow$ aAB $\Rightarrow$ aaBBB$\Rightarrow^{*}$ aaaaa ist die kürzeste. Jede Ableitung fügt ein b hinzu. Somit gibt es 5 Ableitunge kürzer als 11
	\item Nein, Nein, Ja, Nein
	\item
	\end{enumerate}
\subsubsection{Aufgabe 10}
	\begin{enumerate}
		\item Diese ist regulär. Betrachte die reguläre Grammatik G = \{ S $\rightarrow$ bS|aS', S' $\rightarrow$ bS'|aS''| a, S''$\rightarrow$ aS''| bS''| a | b\}
		\item Ist kontextfrei, da sie der Gramamtik a$^{n}$ b$^{n}$
	\end{enumerate}
	
\subsubsection{Aufgabe 11}
\begin{enumerate}
	\item Nein, denn abbaa $\notin$ L, aber a$\varepsilon$a $\in$ L. Nein da Synt verfeinerung von Myhill-Neurode
	\item Klassen $varepsilon$, a, b, bba // gibt es mehr?
	\item M' = (Q, 
\end{enumerate}

\subsection{Theo-Ka 18/19}
	\subsubsection{Aufgabe 1}
	\begin{enumerate}
		\item \( (a|b)^{*}  \)
	\end{enumerate}
	
\section{Theo I}

\subsection{Theo-Ka 16/17}

\subsubsection{Aufgabe 1}
	\begin{enumerate}
		\item bbb ist ein synchronisierendes Wort
		\item 
	\end{enumerate}
	
\subsubsection{Aufgabe 2}
	Der minimale Automat hat 8 Zustände (siehe Hausaufgaben)

\subsubsection{Aufgabe 3}
	\begin{enumerate}
		\item Die betrachtete Sprache ist unär. Somit ist nur zu zeigen, dass sie nicht Typ 3 ist, um zu zeigen dass sie nicht Typ 2 ist.\\
			Angenommen L$_1$ ist regulär. Dann ist L$_1$ unter komplement abgeschlossen. Somit betrachten wir x = uvw = a$^{n}$ n $\in\mathbb{P}$, v = a$^{k}$ k > 0.\\
		\(|uv^{n + 1}w| = |uvw| + n*k = n*(k+1) \notin \bar L_1 \Rightarrow \) $\bar L_1$ nicht regulär \\
		$\Rightarrow L_1$ nicht regulär $\Rightarrow$ L$_1$ nicht kontextfrei
				
		\item  Angenommen L$_2$ wäre Typ 2. Typ 2 Sprachen sind abgeschlossen unter Schnitt mit Typ 3. Wir schneiden mit der Sprache a$^{*}$. Dann erhalten wir L$_1$. Jedoch ist bewiesen, das diese nicht Typ 2 ist $\Rightarrow$ L$_2$ nicht Typ 2
	\end{enumerate}

\subsubsection{Aufgabe 4}
	\begin{enumerate}
	\item 
	\begin{table}[!h]	
	\centering
 	 \begin{tabular}{c|c|c|c|c|c|c|c}	 
       & \circlearound{q\textsubscript{0}}  & \circlearound{q\textsubscript{1}} & \circlearound{q\textsubscript{2}} & \circlearound{q\textsubscript{3}} & \circlearound{q\textsubscript{4}} & \circlearound{q\textsubscript{5}} \\  \hline 	
       
    \circlearound{q\textsubscript{6}} & baa & ab & ba & a & a & $\varepsilon$ \\ \cline{1-7}
    \circlearound{q\textsubscript{5}} & $\varepsilon$ & $\varepsilon$ & $\varepsilon$ & $\varepsilon$ & $\varepsilon$ \\ \cline{1-6}
    \circlearound{q\textsubscript{4}} & baa & ba & ba & X \\ \cline{1-5}
    \circlearound{q\textsubscript{3}} & baa & ba & ba \\ \cline{1-4}
    \circlearound{q\textsubscript{2}} & bba & X \\ \cline{1-3}
    \circlearound{q\textsubscript{1}} & bba \\ \cline{1-2}

	\end{tabular}	
	\end{table}
	\end{enumerate}
	Nach der Tabelle verschmelzen q3 q4 und q2 q1.  
\subsubsection{Aufgabe 5}
	\begin{enumerate}
		\item regulär (abgeschlossenheit unter Konkatenation)
		\item regulär (Abgeschlossen unter allen Operationen)
		\item Typ 2 (da Beispiel Sprache)
		\item Typ 1 (da ein Kellerautomat nur ein Keller besitzt)
		\item Typ 1 (Schnitt mit reg. Sprache  b$^*$ ergibt unäre nicht Typ 2 Sprache)
		\item Typ reg Typ 2
		\item reg Typ 2
		\item Typ 2
	\end{enumerate}
\end{document}
