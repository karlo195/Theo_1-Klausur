
\documentclass[12pt]{scrartcl}

% LaTeX Template für Abgaben an der Universität Stuttgart
% Autor: Sandro Speth
% Bei Fragen: Sandro.Speth@studi.informatik.uni-stuttgart.de
%-----------------------------------------------------------
% Modul fuer verwendete Pakete.
% Neue Pakete einfach einfuegen mit dem \usepackage Befehl:
% \usepackage[options]{packagename}
\usepackage[utf8]{inputenc}
\usepackage[T1]{fontenc}
\usepackage[ngerman]{babel}
\usepackage{lmodern}
\usepackage{graphicx}
\usepackage[pdftex,hyperref,dvipsnames]{xcolor}
\usepackage{listings}
\usepackage[a4paper,lmargin={2cm},rmargin={2cm},tmargin={3.5cm},bmargin = {2.5cm},headheight = {4cm}]{geometry}
\usepackage{amsmath,amssymb,amstext,amsthm}
\usepackage[lined,algonl,boxed]{algorithm2e}
% alternative zu algorithm2e:
%\usepackage[]{algorithm} %counter mit chapter
%\usepackage{algpseudocode}
\usepackage{tikz}
\usepackage{hyperref}
\usepackage{url}
\usepackage[inline]{enumitem} % Ermöglicht ändern der enum Item Zahlen
\usepackage[headsepline]{scrlayer-scrpage} 
\pagestyle{scrheadings} 
\usetikzlibrary{automata,positioning}


\begin{document}
	
\section{Theo I}
\subsection{Theo-Ka 17/18}
	
	
\subsubsection{Aufgabe 6}
	\begin{enumerate}
		\item Die Sprache ist regulär, da \(L_1 = \{w | w \in a^{+} \lor w \in b^{+} \} \)
		\item Die Sprache ist regulär, da \( L_2 = \{c^{*} \} \)
		\item Die Sprache ist det kontextfrei da der kellerautomat nicht raten muss 
		\item Die Sprache ist Typ 1 da sie \( a^{n} b^{n} c^{n} \) ähnelt
		\item Die Sprache ist Typ 2, da sie \(a^{n} b^{n} \) ähnelt jedoch muss sie raten, ob sie die gleichheit von a's und  bs's oder die Gleichheit von a's und c's überprüfen muss
		\item Die Sprache ist Typ 1, da ein endlicher nicht- deterministische Automat diese erkennen kann (wie ww)
		\item Diese ist det. kontextfrei (vgl a$^{n}$b$^{n}$)
		\item Diese ist Typ 2, da Typ 2 (a$^{n}$b$^{n}$ a$^{m}$b$^{m}$)) erkennen kann, aber raten muss wo b$^{m}$ anfängt
		\item Diese ist Typ 1 da unär und nicht Typ 3
		\item Diese ist Typ 3
	\end{enumerate}

\subsubsection{Aufgabe 7}
	\begin{enumerate}
		\item G = \(\{S \rightarrow S'abba | ,\ S' \rightarrow S'AS'BS'| S'BS'AS'|\varepsilon \}\)
	\end{enumerate}

\subsubsection{Aufgabe 8}
	\begin{enumerate}
		\item falsch, wahr, wahr, \(L_3=\{ b^{m+2} a^{n+1}| m,n \in \mathbb{N}_0 \}  \)
		\item Sei das Alphabet $\Sigma$ = \(\{a_1, a_2, ..., a_n  \}\) n $\in$ $\mathbb{N}\ge 1$. Wir ersetzen jedes Terminal a$_n$ in der Grammatik G mit einer Variable A$_n$ und Vereinigen diese mit\(\{ (a_k,A_k A_k),(A_k,a_k)| 0 < k < n \}\). Die resultierende ist G'.
	\end{enumerate}
	
\subsubsection{Aufgabe 9}
	\begin{enumerate}
		\item Ja, Nein, Nein, Nein
		\item 3
		\item Ja, Nein, Nein, Nein
		\item S $\Rightarrow$ aAB $\Rightarrow$ aaBBB$\Rightarrow^{*}$ aaaaa ist die kürzeste Ableitung \{aaaaa, abaaaaaa\}
	\end{enumerate}
\end{document}
